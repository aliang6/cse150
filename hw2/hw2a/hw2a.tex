\documentclass[12pt]{article}
\usepackage[left=0.5in, right=0.5in, top=0.5in, bottom=0.5in]{geometry}
\usepackage{ulem}
\usepackage{amsmath}
\usepackage{amssymb}
\usepackage{enumerate}

\begin{document}
\begin{flushleft}
Andy Liang

CSE 150 - Foundations of Computer Science: Honors

Professor Bender
\end{flushleft}
\medskip
\centerline{\uline{Homework 2A}}
\bigskip\bigskip

\noindent
\uline{Problem 1}
\\*\textbf{For each of the following statements about sets determine whether it is always true (also provide an example), or only sometimes true (also provide an example and counterexample). Please provide an explanation.}
\begin{enumerate}
\item A $\in$ P(A)
\smallskip
\\*\uline{Response}
\\* \textbf{Always True.} The power set of a set is, by definition, P(S) = $\{A | A \subseteq S\}$. Since A is always a subset of itself, it is among the elements of P(S), which makes the statement $A \in P(A)$ always true. 
\\*\uline{Example:}
\\* A: $\varnothing$
\\* P(A): $\{\varnothing\}$
\smallskip
\item A $\subseteq$ P(A)
\smallskip
\\*\uline{Response}
\\* \textbf{Sometimes True.} The definition of a subset is $A \subseteq B \iff (\forall x, x \in A \implies x in B).$ Since the power set of A is composed of \textit{sets} that are created from the \textit{elements} of A, they are not always comparable. 
\\*\uline{Example:}
\\* A: $\{\varnothing\}$
\\* P(A): $\{\varnothing \{\varnothing\}\}$
\\*\uline{Counterexample:}
\\* A: $\{1\}$
\\* P(A): $\{ \{1\} \}$
\smallskip
\item $(|A| \leq |B|) \implies (A \subseteq B)$
\smallskip
\\*\uline{Response}
\\* \textbf{Sometimes True.} There can be less elements in set A than in set B, but that does not imply that the elements in set A are the same as set B. This statement is false if set A holds less elements than set B ($|A| \leq |B|$) and is disjoint from set B ($A \cap B = \varnothing$).
\smallskip
\\*\uline{Example:}
\\* A: $\varnothing$ 
\\* B: \{1\}
\smallskip
\\*\uline{Counterexample:}
\\* A: \{1\}\quad
\\* B: \{2, 3\}\quad
\item $(A \subseteq B) \implies (|A| \leq |B|)$
\smallskip
\\*\uline{Response}
\\* \textbf{Always True.} The definition of a subset is $A \subseteq B \iff (\forall x, x \in A \implies x in B).$ Every element of set A has to be an element of set B, meaning that the amount of elements in set A is either less than or equal to the amount of elements in set B.
\smallskip
\\*\uline{Example:}
\\* A: $\varnothing$\quad$|A| = 0$
\\* B: $\varnothing$\quad$|B| = 0$
\smallskip
\end{enumerate}
\medskip
\uline{Problem 2}
\\*\textbf{Find the smallest two finite sets A and B for each of the following four conditions
\\*\textit{Note:} The smallest sets may not be unique.}
\begin{enumerate}
\item $A \in B, A \subseteq B,$ and $P(A) \subseteq B$
\smallskip
\\*\uline{Example:}
\\* A: $\{\varnothing\}$
\\* B: $\{\varnothing \{\varnothing\}\}$
\smallskip
\item $(\mathbb{N} \cap A) \in A, B \subset A,$ and $P(B) \subseteq A$
\smallskip
\\*\uline{Example:}
\\* A: $\{\varnothing\}$
\\* B: $\varnothing$
\smallskip
\item $A \subseteq (P(P(B)) - P(A))$
\smallskip
\\*\uline{Example:}
\\* A: $\varnothing$
\\* B: $\varnothing$
\smallskip
\item $A \supseteq (P(P(B))-P(A))$
\smallskip
\\*\uline{Example:}
\\* A: $\{\varnothing\}$
\\* B: $\varnothing$
\smallskip
\end{enumerate}
\medskip
\uline{Problem 3}
\\*\textbf{Prove or disprove (by providing a counterexample) each of the following properties of binary relations:
\\*Let S(A) be the symmetric closure of set A. Let T(A) be the transitive closure of set A.
\\* For every binary relation R,}
\begin{enumerate}
\item T(S(R)) $\subseteq$ S(T(R))
\smallskip
\\*\uline{Response}
\\*
\item S(T(R)) $\subseteq$ T(S(R))
\smallskip
\\*\uline{Response}
\\*
\end{enumerate}
\medskip
\uline{Problem 4}
\\*\textbf{How many reflextive binary relations are there on S x S? How many symmetric relations? Explain.}
\\*\textit{Bonus:} How many equivalence relations are there on S x S? Explain.
\smallskip
\\*\uline{Response}
\\*
\end{document}