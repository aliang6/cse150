\documentclass[11pt]{article}
\usepackage[left=0.5in, right=0.5in, top=0.5in, bottom=0.5in]{geometry}
\usepackage{ulem}

\begin{document}

\begin{flushleft}
Andy Liang

CSE 150 - Foundations of Computer Science: Honors

Professor Bender
\end{flushleft}
\medskip
\centerline{\uline{Academic Honesty}}
\bigskip\bigskip

\noindent
\uline{Problem 1}
\\*\textbf{Explain why we let students work together to solve problems, as long as the students cite their
collaborators. Explain why working together is not academically dishonest in this course.}
\medskip
\\*\uline{Response}
\\*There are occasionally times when a student is stuck on a problem because he make the mistake of looking at it the same way instead of trying to approach it in a different way. With collaborators, students have access to different ways of thinking and different perspectives on a problem. The variety of thinking can help advance and expand ideas or help create new approaches to a problem. Working with collaborators is not academically dishonest because the group goes on the same journey together to help each other solve the same problem. They build on each others ideas and help each other learn until they all reach the common goal of figuring out a solution. However, it is important for the student to cite the collaborators because not all of the thinking that was done towards finding the solution was the student's. He needs to credit those that also contributed their perspectives and intellect for their collaboration to be academically honest.
\bigskip\bigskip

\noindent
\uline{Problem 2}
\\*\textbf{Explain why it is important to your professional development to struggle with a problem that you
cannot solve quickly. In other words, the instructor deliberately assigns homework he knows you
will likely have to think about for days or weeks to solve. What do you expect to learn from this
experience?}
\medskip
\\*\uline{Response}
\\*Computer science, like many other skills, is learned through the journey. A problem with a simple solution can have many complex steps that lead up to it. When a student is given a problem that he cannot solve immediately, giving him the solution would not benefit his learning of the subject because he has not struggled with different approaches to understand the reasoning and concept behind the solution. As frustrating as experimenting with different approaches can be, it helps students understand why certain approaches don't work and encourages outisde-the-box thinking. Creativity is at the forefront of research and a professor assigning homework that encourages this type of thinking would help the student get a taste of professional advancement. 

\bigskip\bigskip

\noindent
\uline{Problem 3}
\\*\textbf{Explain why, although it is ok to work with other students, it is plagiarism to share and/or copy other write-ups.  Give an example of collaboration that is academically honest. Give another example of collaboration that is academically dishonest.}
\medskip
\\*\uline{Response}
\\*Working together is academically honest because the students work together to figure out the solution of the problem, citing their collaborators at the end of it. Plagiarism, however, is academically dishonest because the student who is copying another's writeup is getting a direct, streamlined solution that does not show any of the prior errors or mistakes that the student who gave the writeup had to experience and fix. The plagiarizer also usually does not understand the work that he's copying and it is academically dishonest for a student to submit work that he doesn't understand. An example of academically honest collaboration are all students assigning a time to meet up during the day so that they can work together to dissect and process a problem before they attempt to tackle the solution which they will do by sharing potential approaches and each testing it in their own individual way before sharing their results and work. Academically dishonest collaboration is when students get together to share solutions that they found on the internet instead of going through the effort to figure it out themselves. 
\bigskip\bigskip

\noindent
\uline{Problem 4}
\\*\textbf{Explain why it is academically dishonest to share your solution set with another student.  Explain how you could get burned from just sharing your writeup even if you do not copy yourself.}
\medskip
\\*\uline{Response}
\\*Computer science, like many skills, is learned. The struggles involved in solving the problem sets given in class aid in this learning. When a student who has went through the journey of understanding the problem to reach a solution shares his work with others, he gives permission for others to copy and plagiarize his work. He impedes the learning of his peers and can suffer repercussions if the student he is sharing it with does not cite him and instead claims the work as their own. When this happens the teacher would not be able to distinguish who the actual procrastinator is, leading to potential punishments for the student who thought no wrong could come from sharing his work. 
\bigskip\bigskip

\noindent
\uline{Problem 5}
\\*\textbf{Explain why copying (or approximately copying) solutions from the web (or another source) is
plagiarism, even if you cite your source.}
\medskip
\\*\uline{Response}
\\*When a student copies solutions of a problem from the internet, it is considered plagiarism because he is submitting work that he may not understand and, more importantly, has not put any effort into. Even when the student cites the source, he is still submitting work that was not a result of his own thinking or experimentation. Not only do the student's actions constitute plagiarism and academic honesty, he also does not understand the techniques to solve the problem because he has not attempted to solve it himself. 
\bigskip\bigskip

\noindent
\uline{Problem 6}
\\*\textbf{Explain why it is better for your grade to leave a question blank, rather than search for answers on the web.  (Hint: calculate approximately how much a homework problem is worth to your raw score versus an exam question. Feel free to include the risk-benefit analysis of getting caught.)}
\medskip
\\*\uline{Response}
\\* Leaving a question blank indicates to the professor that the student doesn't understand the question. It allows the professor to know what the student knows and what he doesn't know so that he is better able to help the student in understanding the particular topics that the student is having difficulty with. This aids the student's learning and helps him during tests which require thinking and a good understanding of the professor's teaching. However, when the student searches for the solution online, he does not give his professor the same understanding of him, so he does not get the same help. Gradewise, leaving a question blank, even leaving all your homework blank, would result in a maximum of 15 percent being deducted from your grade due to homework. This is very small compared to the 85 percent of your grade that your midterm and final constitutes. Understanding during a test is much more important than just having answers for homework. 
\bigskip\bigskip

\noindent
\uline{Problem 7}
\\*\textbf{Imagine that you are employed at a major software company, say Google, Facebook, or Microsoft, and commit code into a product that you copied from a website. Explain the potential risks to both you and the company if this action is discovered by the owners of the code.}
\medskip
\\*\uline{Response}
\\*If I were an software engineer at a major software company and copied code from a website into a product, I would suffer major repercussions if my plagiarism was discovered. The company would have to suffer public embarrassment from the public and the press as it would have to figure out how to explain my blunder. The company would also risk being sued by the website and company that I have copied the code from resulting in huge financial punishments. I would also probably be fired from the company and have a label put on me about my dishonesty, serving as a red flag to other software companies and resulting in the possible end of my career. 
\bigskip\bigskip

\noindent
\uline{Problem 8}
\\*\textbf{Please speculate on why we decided to make a problem set on academic honesty.}
\medskip
\\*\uline{Response}
\\*Children learn from what they see their parents and other people do. They don't understand right or wrong until the words are defined to them and the values instilled upon them. College freshmen students, although not as innocent as children, think in a similar way. They're entering a new environment with a newfound sense of freedom, but also one of uncertainty. As a result, they experiment and test their academic and personal limits and boundaries, so they can adjust to college life. However, sometimes during this transition period, they don't realize that something they may think is fine, is, in actuality, morally and academically wrong. This is especially true for those who have been committing the fault for years without any repercussion. Starting the with semester with a problem set on academic honesty would set the moral guidelines for success not only academically, but throughout life as it would redefine and correct what some people may have originally though was right and wrong. 
\bigskip\bigskip

\noindent
\uline{Problem 9}
\\*\textbf{How much time did this writeup take you, including the time it took to learn latex?}
\medskip
\\*\uline{Response}
\\*It took me two hours to learn the basics of Latex through online resources and from looking at honesty.tex and it took me an additional five hours to answer the questions.
\bigskip\bigskip

\end{document}