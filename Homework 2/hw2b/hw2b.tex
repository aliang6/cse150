\documentclass[12pt]{article}
\usepackage[left=0.5in, right=0.5in, top=0.5in, bottom=0.5in]{geometry}
\usepackage{ulem}
\usepackage{amsmath}
\usepackage{amssymb}
\usepackage{enumerate}

\begin{document}
\begin{flushleft}
Andy Liang

CSE 150 - Foundations of Computer Science: Honors

Professor Bender
\end{flushleft}
\medskip
\centerline{\uline{Homework 2B}}
\bigskip\bigskip

\noindent
\uline{Problem 5}
\\*\textbf{ Show that the function $f: \mathbb{N} \rightarrow \mathbb{N}$ has the listed properties:}
\begin{enumerate}
\item $f(x) = 2x$ \qquad (one-to-one but not onto)
\smallskip
\\*\uline{Response}
\\*
\item $f(x) = x + 1$ \qquad (one-to-one but not onto)
\smallskip
\\*\uline{Response}
\\*
\item $f(x) =$ if $x$ is odd then $x$ - 1, lse $x$ + 1 \qquad (bijective)
\smallskip
\\*\uline{Response}
\\*
\end{enumerate}
\medskip
\uline{Problem 6}
\\*\textbf{Show that the product $(a + bi) (c + di)$ of two complex numbers can be evaluated using just three real-number multiplications. You may use a few extra additions}
\smallskip
\\*\uline{Response}
\\* 
\medskip
\\*\uline{Problem 7}
\\*\textbf{Given a fixed function $f: A \to A.$ An element $a \in A$ is called a $fixed point$ of $f$ if $f(a) = a.$ Find the set of fixed points for each of the following functions.}
\begin{enumerate}
\item $f: A \to A$ where $f(x) = x$
\smallskip
\\*\uline{Response}
\\*
\item $f: \mathbb{N} \to \mathbb{N}$ where $f(x) = x + 1$
\smallskip
\\*\uline{Response}
\\*
\item $f: \mathbb{N}_6 \to \mathbb{N}_6$ where $f(x) = 2x$ mod 6
\smallskip
\\*\uline{Response}
\\*
\item $f: \mathbb{N}_6 \to \mathbb{N}_6$ where $f(x) = 3x$ mod 6
\smallskip
\\*\uline{Response}
\\*
\end{enumerate}
\medskip
\uline{Problem 8}
\\*\textbf{Let $f(x) = x^2$ and $g(x,y_ = x + y.$ Find compositions that use the functions $f$ and $g$ for each of the following expressions.}
\begin{enumerate}
\item $(x+y)^2$
\smallskip
\\*\uline{Response}
\\*
\item $x^2 + y^2$
\smallskip
\\*\uline{Response}
\\*
\item $(x+y+z)^2$
\smallskip
\\*\uline{Response}
\\*
\item $x^2+y^2+z^2$
\smallskip
\\*\uline{Response}
\\*
\end{enumerate}
\end{document}